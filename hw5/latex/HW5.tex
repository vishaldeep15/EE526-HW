\documentclass[letter, 11pt]{article}
\usepackage{comment} % enables the use of multi-line comments (\ifx \fi) 
\usepackage{lipsum} %This package just generates Lorem Ipsum filler text. 
\usepackage{fullpage} % changes the margin
\usepackage{listings} % code formatting
\usepackage{enumitem} % change enumeration counter
\usepackage{amsmath}
\usepackage{graphicx}
\usepackage{cleveref}
\usepackage{pythonhighlight}
\usepackage{listings}
\usepackage{float}

\lstset{
	basicstyle=\ttfamily,
	columns=fullflexible,
	frame=single,
	breaklines=true,
}

\begin{document}
	
\begin{titlepage}
	
	\newcommand{\HRule}{\rule{\linewidth}{0.5mm}} % Defines a new command for the horizontal lines, change thickness here
	
	\center % Center everything on the page
	
	%----------------------------------------------------------------------------------------
	%	HEADING SECTIONS
	%----------------------------------------------------------------------------------------
	
	\textsc{\LARGE Iowa State University}\\[1.5cm] % Name of your university/college
	\textsc{\LARGE Department of Electrical and Computer Engineering}\\[1.0cm]
	\textsc{\Large Deep Machine Learning: Theory and Practice}\\[0.5cm] % Major heading such as course name
	\textsc{\large EE 526X}\\[0.5cm] % Minor heading such as course title
	
	%----------------------------------------------------------------------------------------
	%	TITLE SECTION
	%----------------------------------------------------------------------------------------
	
	\HRule \\[0.4cm]
	{ \huge \bfseries Homework 5}\\[0.4cm] % Title of your document
	\HRule \\[1.5cm]
	
	%----------------------------------------------------------------------------------------
	%	AUTHOR SECTION
	%----------------------------------------------------------------------------------------
	
	\begin{minipage}{0.4\textwidth}
		\begin{flushleft} \large
			\emph{Author:}\\
			Vishal \textsc{Deep} % Your name
		\end{flushleft}
	\end{minipage}
	~
	\begin{minipage}{0.4\textwidth}
		\begin{flushright} \large
			\emph{Instructor:} \\
			Dr. Zhengdao \textsc{Wang} % Supervisor's Name
		\end{flushright}
	\end{minipage}\\[2cm]
	
	% If you don't want a supervisor, uncomment the two lines below and remove the section above
	%\Large \emph{Author:}\\
	%John \textsc{Smith}\\[3cm] % Your name
	
	%----------------------------------------------------------------------------------------
	%	DATE SECTION
	%----------------------------------------------------------------------------------------
	
	{\large \today}\\[2cm] % Date, change the \today to a set date if you want to be precise
	
	%----------------------------------------------------------------------------------------
	%	LOGO SECTION
	%----------------------------------------------------------------------------------------
	\includegraphics[scale=0.4]{isu.PNG}\\[1cm] % Include a department/university logo - this will require the graphicx package
	
	%----------------------------------------------------------------------------------------
	
	\vfill % Fill the rest of the page with whitespace
	
\end{titlepage}
%Header-Make sure you update this information!!!!
%\noindent
%\large\textbf{Homework21} \hfill \textbf{Vishal Deep} \\
%\normalsize CPRE 581  \hfill ID: 010639648 \\
%Dr. Zhengdao  \hfill Due Date: 09/05/18 \\


\section{Problem 1}

%\inputpython{../cnn.py}{1}{124}

%\begin{lstlisting}[language=Python, basicstyle=\tiny]
%
%\end{lstlisting}



\begin{figure}[H]
	\centering
	\includegraphics[scale=0.3]{nw-graph.png}
	\caption{Network Graph}
\end{figure}


\section{Problem 2}
States: S = {0, 1} \\
Actions: A = {1, 2} \\
Rewards:
$$ R_{S}^{(a)} =
 \begin{cases} 
      1 & (s,a) = (0,1) \\
      4 & (s,a) = (0,2) \\
      3 & (s,a) = (1,1) \\
      2 & (s,a) = (1,2)
   \end{cases}
$$ \\
Transition Probabilities:
$$
\begin{bmatrix}
P_{00}^{(1)} & P_{00}^{(2)} \\
P_{10}^{(1)} & P_{10}^{(2)} 
\end{bmatrix}	= 
\begin{bmatrix}
	\frac{1}{3} & \frac{1}{2} \\
	\frac{1}{4} & \frac{2}{3}
\end{bmatrix}
$$

Discount factor: $ \gamma = \frac{3}{4}$
	
The Bellman's expectation equation is given by

$$ V_{\pi}(S) = R_{S} + \gamma \sum_{s^{'} \epsilon S} P_{SS^{'}} V_{\pi}(S^{'}) $$

\subsection*{2(a):}
choosing action 1 in state 0, and action 2 in state 1,

$$ V_{\pi}(0) = R_{0}^{(1)} + \gamma (P_{00}^{(1)} V_{\pi}(0) + P_{01}^{(1)} V_{\pi}(1)) $$
$$ V_{\pi}(1) = R_{1}^{(2)} + \gamma (P_{10}^{(2)} V_{\pi}(0) + P_{11}^{(2)} V_{\pi}(1)) $$

Substituting the values,
$$ V_{\pi}(0) = 1 + \frac{3}{4} (\frac{1}{3} V_{\pi}(0) + \frac{2}{3} V_{\pi}(1)) $$
$$ V_{\pi}(1) = 2 + \frac{3}{4} (\frac{2}{3} V_{\pi}(0) + \frac{1}{3} V_{\pi}(1)) $$

Solving these 2 equations, we get
$$ V_{\pi}(0) = \frac{28}{5} $$
$$ V_{\pi}(1) = \frac{32}{5} $$

\subsection*{2(b):}
\inputpython{../prob2-b.py}{1}{11}

\begin{lstlisting}[language=Python, basicstyle=\tiny]
Iteration = 0, vpi_0 = 1.0, vpi_1 = 2.0
Iteration = 1, vpi_0 = 2.25, vpi_1 = 3.0
Iteration = 2, vpi_0 = 3.0625, vpi_1 = 3.875
Iteration = 3, vpi_0 = 3.7031, vpi_1 = 4.5
Iteration = 4, vpi_0 = 4.17578125, vpi_1 = 4.9765625
\end{lstlisting}

\subsection*{2(c):}
The Bellman's expectation equation for $ q_{\pi}(s, a) $ is

$$ q_{\pi}(s, a) = R_{S}^{(a)} + \gamma \sum_{s^{'} \epsilon S} P_{SS^{'}}^{(a)} V_{\pi}(S^{'}) $$

$$ q_{\pi}(0, 1) = R_{0}^{(1)} + \gamma (P_{00}^{(1)} V_{\pi}(0) + P_{01}^{(1)} V_{\pi}(1)) $$

$$ q_{\pi}(0, 1) = 1 + \frac{3}{4} (\frac{1}{3} V_{\pi}(0) + \frac{2}{3} V_{\pi}(1)) = \frac{28}{5} $$

$$ q_{\pi}(0, 2) = R_{0}^{(2)} + \gamma (P_{00}^{(2)} V_{\pi}(0) + P_{01}^{(2)} V_{\pi}(1)) $$

$$ q_{\pi}(0, 2) = 4 + \frac{3}{4} (\frac{1}{2} V_{\pi}(0) + \frac{1}{2} V_{\pi}(1)) = \frac{17}{2} $$

$$ q_{\pi}(1, 1) = R_{1}^{(1)} + \gamma (P_{10}^{(1)} V_{\pi}(0) + P_{11}^{(1)} V_{\pi}(1)) $$

$$ q_{\pi}(1, 1) = 3 + \frac{3}{4} (\frac{1}{4} V_{\pi}(0) + \frac{3}{4} V_{\pi}(1)) = \frac{153}{20} $$

$$ q_{\pi}(1, 2) = R_{1}^{(2)} + \gamma (P_{10}^{(2)} V_{\pi}(0) + P_{11}^{(2)} V_{\pi}(1)) $$ 

$$ q_{\pi}(1, 2) = 4 + \frac{3}{4} (\frac{1}{2} V_{\pi}(0) + \frac{1}{2} V_{\pi}(1)) = \frac{32}{5} $$

\subsection*{2(d)}

From the part c, \\

$$ State 0: q_{\pi}(0, 1) < q_{\pi}(0, 2) $$
$$ State 1: q_{\pi}(1, 1) > q_{\pi}(1, 2) $$

Improved policy would be to choose action 2 in state 0 and action 1 in state 1.

\subsection*{2(e)}
\inputpython{../prob2-ef.py}{1}{44}

\begin{lstlisting}[language=Python, basicstyle=\tiny]
Iteration: 0, V0: 4.0000, V1: 3.0000
Iteration: 1, V0: 6.6250, V1: 5.4375
Iteration: 2, V0: 8.5234, V1: 7.3008
Iteration: 3, V0: 9.9341, V1: 8.7048
Iteration: 4, V0: 10.9896, V1: 9.7591
Iteration: 5, V0: 11.7808, V1: 10.5500
Iteration: 6, V0: 12.3741, V1: 11.1433
Iteration: 7, V0: 12.8190, V1: 11.5882
Iteration: 8, V0: 13.1527, V1: 11.9219
Iteration: 9, V0: 13.4030, V1: 12.1722
q(0,1) = 10.4369
q(0,2) = 13.5907
q(1,1) = 12.3599
q(1,2) = 11.7446
\end{lstlisting}

\subsection*{2(f)}

As calculated from the code above, Optimal Policy is

$$ 	q(0,1) = 10.4369 $$
$$	q(0,2) = 13.5907 $$
$$	q(1,1) = 12.3599 $$
$$	q(1,2) = 11.7446 $$

\section*{Problem 3}

\end{document}
