\documentclass[letterpaper , 11pt]{article}
\usepackage{comment} % enables the use of multi-line comments (\ifx \fi) 
\usepackage{lipsum} %This package just generates Lorem Ipsum filler text. 
\usepackage{fullpage} % changes the margin
\usepackage{listings} % code formatting
\usepackage{enumitem} % change enumeration counter
\usepackage{amsmath}
\usepackage{graphicx}

%\usepackage{pythonhighlight}
\usepackage{listings}
\usepackage{hyperref}
\usepackage{cleveref}
\usepackage{float}
\usepackage{algorithm}
\usepackage{algpseudocode}

\lstset{
	basicstyle=\ttfamily,
	columns=fullflexible,
	frame=single,
	breaklines=true,
}

\begin{document}
	
\begin{titlepage}
	
	\newcommand{\HRule}{\rule{\linewidth}{0.5mm}} % Defines a new command for the horizontal lines, change thickness here
	
	\center % Center everything on the page
	
	%----------------------------------------------------------------------------------------
	%	HEADING SECTIONS
	%----------------------------------------------------------------------------------------
	
	\textsc{\LARGE Iowa State University}\\[1.5cm] % Name of your university/college
	\textsc{\LARGE Department of Electrical and Computer Engineering}\\[1.0cm]
	\textsc{\Large Deep Machine Learning: Theory and Practice}\\[0.5cm] % Major heading such as course name
	\textsc{\large EE 526X}\\[0.5cm] % Minor heading such as course title
	
	%----------------------------------------------------------------------------------------
	%	TITLE SECTION
	%----------------------------------------------------------------------------------------
	
	\HRule \\[0.4cm]
	{ \huge \bfseries Final Project Report}\\[0.4cm] % Title of your document
	\HRule \\[1.5cm]
	
	%----------------------------------------------------------------------------------------
	%	AUTHOR SECTION
	%----------------------------------------------------------------------------------------
	
	\begin{minipage}{0.4\textwidth}
		\begin{flushleft} \large
			\emph{Author:}\\
			Vishal \textsc{Deep} % Your name
		\end{flushleft}
	\end{minipage}
	~
	\begin{minipage}{0.4\textwidth}
		\begin{flushright} \large
			\emph{Instructor:} \\
			Dr. Zhengdao \textsc{Wang} % Supervisor's Name
		\end{flushright}
	\end{minipage}\\[2cm]
	
	% If you don't want a supervisor, uncomment the two lines below and remove the section above
	%\Large \emph{Author:}\\
	%John \textsc{Smith}\\[3cm] % Your name
	
	%----------------------------------------------------------------------------------------
	%	DATE SECTION
	%----------------------------------------------------------------------------------------
	
	{\large \today}\\[5cm] % Date, change the \today to a set date if you want to be precise
	
	%----------------------------------------------------------------------------------------
	%	LOGO SECTION
	%----------------------------------------------------------------------------------------
	
	\includegraphics[scale=0.3]{isu.png}\\[1cm] % Include a department/university logo - this will require the graphicx package
	
	%----------------------------------------------------------------------------------------
	
	\vfill % Fill the rest of the page with whitespace
	
\end{titlepage}
%Header-Make sure you update this information!!!!
%\noindent
%\large\textbf{Homework21} \hfill \textbf{Vishal Deep} \\
%\normalsize CPRE 581  \hfill ID: 010639648 \\
%Dr. Zhengdao Duwe \hfill Due Date: 09/05/18 \\


\section{Problem Statement}
The main goal of the project is to design and optimize a reinforcement learning algorithm using double Q-learning on Acrobot-v1 environment from Open AI gym package \cite{openAI-gym}. The function approximation should be done with multi-layer neural networks. 

\subsection*{Acrobot-v1}
The Acrobot-v1 environment consists of two joints. One joint is fixed and other joint is actuated as shown in \cref{fig:acrobot}. In the initial stage the joints are hanging downwards and the goal is to make lower joint reach a certain given height. This environment is unsolved environment which means it does not have a specific reward above which we can say that it is solved.

\begin{figure}[H]
	\centering
	\includegraphics[scale=0.5]{acrobot.jpg}
	\caption{Acrobot-v1 from Open AI gym package}
	\label{fig:acrobot}
\end{figure}

\begin{itemize}
	\item \textbf{State}: In this environment states consists of $\sin$ and $\cos$ of the two rotational joint angles and their angular velocities. A complete state is \newline
	[cos(theta1) sin(theta1) cos(theta2) sin(theta2) thetaDot1 thetaDot2].
	\item \textbf{Action}: There are three possible actions in this environment. The action is either applying +1, 0 or -1 torque on the joint between the two links.
\end{itemize}

\subsection*{Deep Q-learning}
In the Q-learning algorithm, a virtual table of action and Q-values is used. This table could grow really fast depending on the states and actions. Therefore a better solution is to create Neural Networks (NNs) which can approximate the Q-values for every action. This is called deep Q-learning. In case of Q-learning, taking the maximum overestimated values introduces a maximization bias in learning.  To solve this problem, we use a double Q-learning algorithm which uses two separate Q-value estimators. These estimators are used to update each other and thereby providing unbiased estimation of Q-values. In this project, I will be using double Q-learning function estimators using neural networks.

\section{Neural Network Model}
In this project, I have used the neural network shown in \cref{fig:nn}. This NN has three fully connected dense layers. The first layer consists of 48 neurons, the second one has 24 neurons, and last layer consists of 3 neurons to signify the 3 output actions. The total number of parameters are 1587.  

\begin{figure}[H]
	\centering
	\includegraphics[scale=0.8]{model.png}
	\caption{Neural Network Model used for function approximation in double Q-learning algorithm}
	\label{fig:nn}
\end{figure}

\section{Pseudo-code for the Algorithm}
%\begin{algorithm}[H]
%	Initialize primary $Q_{A}$ and target network $Q_{B}$
%	\For{\texttt{each episode}} \Do
%		\For{each step} \Do
%			Choose action $a_t$ which maximize q-value
%			Execute $a_t$ and observe State $S_{t+1}$ and Reward $R_t$
%			Save ($S_{t}, $a_t$, $R_t$, $S_{t+1}$) to memory
%			
%	\EndFor
%	\caption{Double Q-learning Algorithm}
%	\label{algo:doubleQ}
%\end{algorithm}

\begin{algorithm}[H]
	\caption{Euclid’s algorithm}\label{euclid}
	\begin{algorithmic}[1]
		\State Initialize primary $Q_{A}$ and target network $Q_{B}$
		\For{each episode} \Do
		\For{each step} \Do
		\State Choose action $a_t$ which maximize q-value
		\State Execute $a_t$ and observe State $S_{t+1}$ and Reward $R_t$
		\State Save ($S_{t}, $a_t$, $R_t$, $S_{t+1}$) to memory
		\EndFor
		\EndFor
		
		\Procedure{Euclid}{$a,b$}
		\State $r\gets a\bmod b$
		\While{$r\not=0$}\Comment{We have the answer if r is 0}
		\State $a\gets b$
		\State $b\gets r$
		\State $r\gets a\bmod b$
		\EndWhile\label{euclidendwhile}
		\For{\texttt{<some condition>}}
		\State \texttt{<do stuff>}
		\EndFor
		\State \textbf{return} $b$\Comment{The gcd is b}
		\EndProcedure
	\end{algorithmic}
\end{algorithm}


\bibliography{references} 
\bibliographystyle{ieeetr}






\end{document}
