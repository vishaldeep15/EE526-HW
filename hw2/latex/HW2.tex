\documentclass[letter, 11pt]{article}
\usepackage{comment} % enables the use of multi-line comments (\ifx \fi) 
\usepackage{lipsum} %This package just generates Lorem Ipsum filler text. 
\usepackage{fullpage} % changes the margin
\usepackage{listings} % code formatting
\usepackage{enumitem} % change enumeration counter
\usepackage{amsmath}
\usepackage{graphicx}
\usepackage{cleveref}

\lstset{
	basicstyle=\ttfamily,
	columns=fullflexible,
	frame=single,
	breaklines=true,
}

\begin{document}
	
\begin{titlepage}
	
	\newcommand{\HRule}{\rule{\linewidth}{0.5mm}} % Defines a new command for the horizontal lines, change thickness here
	
	\center % Center everything on the page
	
	%----------------------------------------------------------------------------------------
	%	HEADING SECTIONS
	%----------------------------------------------------------------------------------------
	
	\textsc{\LARGE Iowa State University}\\[1.5cm] % Name of your university/college
	\textsc{\LARGE Department of Electrical and Computer Engineering}\\[1.0cm]
	\textsc{\Large Deep Machine Learning: Theory and Practice}\\[0.5cm] % Major heading such as course name
	\textsc{\large EE 526X}\\[0.5cm] % Minor heading such as course title
	
	%----------------------------------------------------------------------------------------
	%	TITLE SECTION
	%----------------------------------------------------------------------------------------
	
	\HRule \\[0.4cm]
	{ \huge \bfseries Homework 2}\\[0.4cm] % Title of your document
	\HRule \\[1.5cm]
	
	%----------------------------------------------------------------------------------------
	%	AUTHOR SECTION
	%----------------------------------------------------------------------------------------
	
	\begin{minipage}{0.4\textwidth}
		\begin{flushleft} \large
			\emph{Author:}\\
			Vishal \textsc{Deep} % Your name
		\end{flushleft}
	\end{minipage}
	~
	\begin{minipage}{0.4\textwidth}
		\begin{flushright} \large
			\emph{Instructor:} \\
			Dr. Zhengdao \textsc{Wang} % Supervisor's Name
		\end{flushright}
	\end{minipage}\\[2cm]
	
	% If you don't want a supervisor, uncomment the two lines below and remove the section above
	%\Large \emph{Author:}\\
	%John \textsc{Smith}\\[3cm] % Your name
	
	%----------------------------------------------------------------------------------------
	%	DATE SECTION
	%----------------------------------------------------------------------------------------
	
	{\large \today}\\[2cm] % Date, change the \today to a set date if you want to be precise
	
	%----------------------------------------------------------------------------------------
	%	LOGO SECTION
	%----------------------------------------------------------------------------------------
	
	\includegraphics[width=0.5\columnwidth]{Iowa_State_Cyclones_logo}\\[1cm] % Include a department/university logo - this will require the graphicx package
	
	%----------------------------------------------------------------------------------------
	
	\vfill % Fill the rest of the page with whitespace
	
\end{titlepage}
%Header-Make sure you update this information!!!!
%\noindent
%\large\textbf{Homework21} \hfill \textbf{Vishal Deep} \\
%\normalsize CPRE 581  \hfill ID: 010639648 \\
%Dr. Zhengdao Duwe \hfill Due Date: 09/05/18 \\

\section{Problem 1}
We have a softplus function,
\begin{equation}
	f(z) = \log(1 + e^{z}) 
\end{equation} 

Taking first derivative of the function:
\begin{equation}
	f'(z) = \frac{1}{(1 + e^{z})}e^{z}	
\end{equation}

multiplying and dividing by $e^{-z}$,
\begin{equation}
	f'(z) = \frac{e^{z}.e^{-z}}{(1.e^{-z} + e^{z}.e^{-z})}
	 	  = \frac{1}{(1 + e^{-z})}
\end{equation}

Now taking second derivative of softplus function,

\begin{equation}
	f''(z) = \frac{(1+e^{-z}).0+e^{-z}}{(1+e^{-z})^{2}}
\end{equation}

\begin{equation}\label{eqn:2ndderivative}
	f''(z) = \frac{e^{-z}}{(1+e^{-z})^{2}}
\end{equation}

\noindent From \cref{eqn:2ndderivative}, numerator is always positive because exponential is a positive number and any power to a positive is number is always a positive number. The denominator is always positive because is squared. Therefore we can say that second derivative of a softplus function is always positive for every value of z. This implies that a softplus function is convex in z.

\section{Problem 2}
We have a vector $ z = [z1, z2, ... , zn]^{T} $ and  $ p = [p1, p2, ... , pn]^{T} $, where p is output after z is applied to a softmax function.

\begin{equation}
	p_{i} = \frac{e^{z_{i}}}{\sum_{j=1}^{n} e^{z_{j}}}
\end{equation}

\noindent The jacobian matrix will have two types of elements diagonal and off-diagonal, we will calculate both and then generalize the result for all elements. \\

\textbf{Diagonal row elements (i=j):}
\begin{align*}
	\frac{\partial{p_i}}{\partial{z_i}} = \frac{\sum_{j=1}^{n}e^{z_{j}}.e^{z_{i}}- e^{z_{i}}.e^{z_{i}}}{(\sum_{j=1}^{n}e^{z_{j}})^2} \\
	\frac{\partial{p_i}}{\partial{z_i}} = \frac{e^{z_{i}}}{\sum_{j=1}^{n}e^{z_{j}}} - \left(\frac{e^{z_{i}}}{\sum_{j=1}^{n}e^{z_{j}}} \right)^2 
\end{align*}
	
\begin{equation}\label{eqn:diagonal}
	\frac{\partial{p_i}}{\partial{z_i}} = p_i - p_{i}^{2}
\end{equation}

\textbf{off-diagonal row elements (i$\neq$j):}
\begin{align*}
	\frac{\partial{p_i}}{\partial{z_j}} = \frac{\sum_{j=1}^{n}e^{z_{j}}.0 - e^{z_{i}}.e^{z_{j}}}{(\sum_{j=1}^{n}e^{z_{j}})^2} \\
	\frac{\partial{p_i}}{\partial{z_j}} = - \frac{e^{z_{i}}}{\sum_{j=1}^{n}e^{z_{j}}} \times \frac{e^{z_{i}}}{\sum_{j=1}^{n}e^{z_{j}}} 
\end{align*}

\begin{equation}\label{eqn:offdiagonal}
	\frac{\partial{p_i}}{\partial{z_j}} = - p_{i} p_{j}
\end{equation}

The Jacobian matrix is given by
\begin{equation}
\frac{\partial{p}}{\partial{z}} = 
\begin{bmatrix}
\frac{\partial{p_{0}}}{\partial{z_{0}}} & \frac{\partial{p_{0}}}{\partial{z_{1}}} & \cdots & \frac{\partial{p_{0}}}{\partial{z_{n}}} \\
\frac{\partial{p_{1}}}{\partial{z_{0}}} & \frac{\partial{p_{1}}}{\partial{z_{1}}} &\cdots & \frac{\partial{p_{1}}}{\partial{z_{n}}} \\
\vdots & \vdots & \ddots & \vdots\\
\frac{\partial{p_{n}}}{\partial{z_{0}}} & \frac{\partial{p_{n}}}{\partial{z_{1}}} &\cdots & \frac{\partial{p_{n}}}{\partial{z_{n}}} \\
\end{bmatrix}
\end{equation}

using \cref{eqn:diagonal} and \cref{eqn:offdiagonal}, we get

\begin{equation}
\frac{\partial{p}}{\partial{z}} = 
\begin{bmatrix}
p_{0} - p_{0}^2 & -p_{0}p_{1} & \cdots & -p_{0}p_{n} \\
-p_{1}p_{0} & p_{1}-p_{1}^2 &\cdots & -p_{1}p_{n} \\
\vdots & \vdots & \ddots & \vdots\\
-p_{n}p_{0} & -p_{n}p_{1} &\cdots &  p_{n}-p_{n}^2\\
\end{bmatrix}
\end{equation}


\section{Problem 3}
We have $ y = [y1, y2, ... , yn]^{T} $, a correct probability vector. And the cross entropy is given by 

\begin{equation}\label{eqn:crossentropy}
	J(z) = - \sum_{i=1}^{n} y_{i}\log{p_i}
\end{equation}

\cref{eqn:crossentropy} is a dot product or element-wise product of $y_i$ and $\log{p_i}$. Taking derivative of \cref{eqn:crossentropy} w.r.t. $p_i$,

\begin{align*}
	\frac{\partial{J}}{\partial{p_i}} = - \frac{\partial{}}{\partial{p_i}}  y_{i} \log{p_i} 
	= - y_{i} \frac{\partial{\log{p_i}}}{\partial{p_i}}    
\end{align*}

\begin{equation}\label{eqn:Jp}
	\frac{\partial{J}}{\partial{p_i}} = - \frac{y_i}{p_i}
\end{equation}

Also from \cref{eqn:diagonal,eqn:offdiagonal} we know $\frac{\partial{p_i}}{\partial{z_j}}$,

\begin{equation}\label{eqn:pz}
	\frac{\partial{p_i}}{\partial{z_j}} =
	\begin{cases} 
      p_i - p_{i}^{2} & i = j\\
     - p_{i} p_{j} & i \neq j
   \end{cases}
\end{equation}

from \cref{eqn:Jp,eqn:pz}, we can find $\frac{\partial{J}}{\partial{z_i}}$
\[
\begin{aligned}
	\frac{\partial{J}}{\partial{z_i}} & = \sum_{j=1}^{n} \frac{\partial{J}}{\partial{p_j}} \frac{\partial{p_j}}{\partial{z_i}} \\
	& = \frac{\partial{J}}{\partial{p_i}} \frac{\partial{p_i}}{\partial{z_i}} + \sum_{i \neq j} \frac{\partial{J}}{\partial{p_j}} \frac{\partial{p_j}}{\partial{z_i}} \\
	& = - \frac{y_i}{p_i} (p_i - p_{i}^{2}) + \sum_{i \neq j} (- \frac{y_j}{p_j}) (- p_{i} p_{j}) \\
	& = - y_{i}(1 - p_{i}) + \sum_{i \neq j} y_{j} p_{i} \\
	& = - y_{i} + y_{i}p_{i} + p_{i}  \sum_{j} y_{j} \\
	& = p_{i} \left(y_{i} + \sum_{j} y_{j} \right) - y_{i} \\
\end{aligned}
\]

Because y is one hot encoded and $\sum_{j}y_{j} = 1$
\begin{equation}
	\frac{\partial{J}}{\partial{z_i}} = p_{i} - y_{i}
\end{equation}








\end{document}
