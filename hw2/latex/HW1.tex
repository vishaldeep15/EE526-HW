\documentclass[letter, 11pt]{article}
\usepackage{comment} % enables the use of multi-line comments (\ifx \fi) 
\usepackage{lipsum} %This package just generates Lorem Ipsum filler text. 
\usepackage{fullpage} % changes the margin
\usepackage{listings} % code formatting
\usepackage{enumitem} % change enumeration counter
\usepackage{amsmath}
\usepackage{graphicx}

\lstset{
	basicstyle=\ttfamily,
	columns=fullflexible,
	frame=single,
	breaklines=true,
}

\begin{document}
	
\begin{titlepage}
	
	\newcommand{\HRule}{\rule{\linewidth}{0.5mm}} % Defines a new command for the horizontal lines, change thickness here
	
	\center % Center everything on the page
	
	%----------------------------------------------------------------------------------------
	%	HEADING SECTIONS
	%----------------------------------------------------------------------------------------
	
	\textsc{\LARGE Iowa State University}\\[1.5cm] % Name of your university/college
	\textsc{\LARGE Department of Electrical and Computer Engineering}\\[1.0cm]
	\textsc{\Large Deep Machine Learning: Theory and Practice}\\[0.5cm] % Major heading such as course name
	\textsc{\large EE 526X}\\[0.5cm] % Minor heading such as course title
	
	%----------------------------------------------------------------------------------------
	%	TITLE SECTION
	%----------------------------------------------------------------------------------------
	
	\HRule \\[0.4cm]
	{ \huge \bfseries Homework 1}\\[0.4cm] % Title of your document
	\HRule \\[1.5cm]
	
	%----------------------------------------------------------------------------------------
	%	AUTHOR SECTION
	%----------------------------------------------------------------------------------------
	
	\begin{minipage}{0.4\textwidth}
		\begin{flushleft} \large
			\emph{Author:}\\
			Vishal \textsc{Deep} % Your name
		\end{flushleft}
	\end{minipage}
	~
	\begin{minipage}{0.4\textwidth}
		\begin{flushright} \large
			\emph{Instructor:} \\
			Dr. Henry \textsc{Duwe} % Supervisor's Name
		\end{flushright}
	\end{minipage}\\[2cm]
	
	% If you don't want a supervisor, uncomment the two lines below and remove the section above
	%\Large \emph{Author:}\\
	%John \textsc{Smith}\\[3cm] % Your name
	
	%----------------------------------------------------------------------------------------
	%	DATE SECTION
	%----------------------------------------------------------------------------------------
	
	{\large \today}\\[2cm] % Date, change the \today to a set date if you want to be precise
	
	%----------------------------------------------------------------------------------------
	%	LOGO SECTION
	%----------------------------------------------------------------------------------------
	
	\includegraphics[width=0.5\columnwidth]{Iowa_State_Cyclones_logo}\\[1cm] % Include a department/university logo - this will require the graphicx package
	
	%----------------------------------------------------------------------------------------
	
	\vfill % Fill the rest of the page with whitespace
	
\end{titlepage}
%Header-Make sure you update this information!!!!
\noindent
\large\textbf{Homework 1} \hfill \textbf{Vishal Deep} \\
\normalsize CPRE 581  \hfill ID: 010639648 \\
Dr. Henry Duwe \hfill Due Date: 09/05/18 \\

\textbf{Note: Problems are completed using 5th edition of the text book.}
\section{Problem 1.1}
\begin{enumerate}[label=(\alph*)]
	\item Die yield can be determined by the following formula: 
	
	$$ Die \,Yield = Wafer\, Yield \times \frac{1}{(1 + (Defects \, per\,unit\, area \times Die \,area)/N)^N} $$
	
	Let us assume wafer yield is 100\% and process complexity 'N' = 10.
	
	$$ Die \,Yield = \frac{1}{(1 + ((0.30 \times 3.89)/10)^{10}} = 0.33 $$
	
	\item The manufacturing size of IBM Power5 is 130nm which is larger than Niagra (90nm) and Opteron (90nm). This implies that Power5 is older hence more matured process. Therefore defects per unit area is small.
\end{enumerate}

\section{Problem 1.2}
\begin{enumerate}[label=(\alph*)]
	\item We know that
	$$ Dies \, per \, wafer = \frac{\pi \times (Wafer \,diameter/2)^2}{Die \, area} - 
		\frac{\pi \times Wafer \, diameter}{\sqrt{2 \times Die \, area}} $$
		$$ Dies \, per \, wafer = \frac{\pi \times (30/2)^2}{1.5} - 
		\frac{\pi \times 30}{\sqrt{2 \times 1.5}} $$
		$$ = 471.24 - 54.48 = 416.76 $$
		Therefore 416 dies can be made from this wafer. To find defect free die, we need to find die yield.
		
		$$ Die \,Yield = \frac{1}{(1 + ((0.30 \times 1.5)/10)^{10}} = 0.64 $$
		
		$$ Profit = 416 \times 0.64 \times \$20 = \$5324.8 $$
		
	\item 
	
	$$ Dies \, per \, wafer = \frac{\pi \times (30/2)^2}{2.5} - 
	\frac{\pi \times 30}{\sqrt{2 \times 2.5}} $$
	$$ = 282.74 - 42.08 = 240.66 $$
	Therefore 240 dies can be made from this wafer. To find defect free die, we need to find die yield.
	
	$$ Die \,Yield = \frac{1}{(1 + ((0.30 \times 2.5)/10)^{10}} = 0.49 $$
	
	$$ Profit = 240 \times 0.49 \times \$25 = \$2940 $$
	
	\item Wood chips generate more profits. Therefore wood ships should be produced in this facility.
	
	\item 
	For 50000 wood chips, I need 50000/416 = 120.19 wafers per month. While for 25000 Markon chips, I need 25000/240 = 104.17 wafers per month.
	By looking at demand and maximum profit, I would make 120 wafers of wood chips and 30 wafers of Markon chips per month.
\end{enumerate}

\section*{Problem 1.3}

\begin{enumerate}[label=(\alph*)]
	\item First we need to find yield of a single core chip
	$$ Die \,Yield = \frac{1}{(1 + ((0.75 \times 1.99)/10)^{10}} =  0.25 $$
	Probability of both core work = 0.25 $\times$ 0.25 = 0.06 \\
	Probability that one core is faulty = 1st faulty $\times$ 2nd good + 1st good $\times$ 2nd faulty 
	= 0.75   $\times$ 0.25 + 0.25  $\times$ 0.75 = 0.38 \\
	Now, probability that a defect will occur on no more than
	one of the two processor cores = 0.06 + 0.38 = 0.44
	
	\item 
	$$ Cost \, of \, die = \frac{Cost \, of \, wafer}{Dies \, per \, wafer \times Die \, yield} $$
	$$ \$20 \times 0.25 = \frac{Cost \, of \, wafer}{Dies \, per \, wafer} $$
	$$ Dies \, per \, wafer = Wafer \, area / Die \, area $$
	For the new chip:
	$$ Cost \, of \, die (new) = \frac{Cost \, of \, wafer}{1/2 \times Dies \, per \, wafer \times 0.44} $$
	$$ = \frac{20 \times 0.25}{1/2 \times  0.44} = \$22.73 $$
\end{enumerate}

\section*{Problem 1.4}
\begin{enumerate}[label=(\alph*)]
	\item Total power consumed by the system on maximum load will be  maximum power consumed by Intel Pentium 4 chip, 2 GB 240-pin Kingston DRAM, and one 7200 rpm hard drive.
	Also, the power supply efficiency of system is 80\%. Therefore,
	
	$$ 0.80 \times Power = 66 + (2 \times 2.3) + 7.9 $$
	$$ Power = 98.12 \, Watts $$
	
	\item From the table, we know that power consumed by disk drive is 7.9 W read/seek, 4.0 W idle. And it is idle 60\% times.
	
	$$ 0.6 \times 4 + 0.4 \times 7.9 = 5.56 W $$
	
	\item Given that
	$$ read7200 = 0.75 \times read5400 $$
	Power consumption of disk drives will be equal when
	$$ read7200 \times 7.9 + idle7200 \times 4 = read5400 \times 7 + idle5400 \times 2.9 $$
	and we also know that 
	$$ read7200 + idle7200 = 1 $$
	$$ read5400 + idle5400 = 1 $$
	Solving these equations for idle7200
	$$ (1-idle7200) \times 7.9 + idle7200 \times 4 = \frac{1-idle7200}{0.75} \times 7 + (1-(\frac{1-idle7200}{0.75})) \times 2.9 $$
	
	$$idle7200 = 0.29 $$
\end{enumerate}

\section*{Problem 3}
\begin{enumerate}[label=(\alph*)]
	\item I logged into 'linux-5.ece.iastate.edu'.
	\item The system used 40 shared CPUs. To find information about CPU, I used following command:
	\begin{lstlisting}[language=bash, xleftmargin=\dimexpr-\leftmargini]
	$ cat /proc/cpuinfo
	\end{lstlisting}
	The CPU is Intel Xeon(R) E5-2660 V3 @ 2.60 GHz. It is a 10 core CPU.
	\item The execution time for the native input and minimum 4 number of threads is 1 min 22.8 sec as shown in figure below. \\
	\includegraphics[width=\textwidth]{result}
	
\end{enumerate}

\subsection*{Transcript to run x264 benchmark on PARSEC} 
This procedure is taken from the PARSEC tutorial \cite{tutorial}.
\begin{enumerate}
	\item Remote login into server \\
	Open terminal and type in
	\begin{lstlisting}[language=bash, xleftmargin=\dimexpr-\leftmargini]
	$ ssh linux-5.ece.iastate.edu
	\end{lstlisting}
	then enter your credentials to log in.
	\item Download PARSEC 3.0 

	Download the PARSEC 3.0 full package using the following command: 
	\begin{lstlisting}[language=bash, xleftmargin=\dimexpr-\leftmargini]
	 $ wget http://parsec.cs.princeton.edu/download/3.0/parsec-3.0.tar.gz
	\end{lstlisting}
	
	\item Unpack downloaded package 
	
	Unpack the downloaded package using following command: 
	\begin{lstlisting}[language=bash, xleftmargin=\dimexpr-\leftmargini]
	$ tar -xzf parsec-3.0.tar.gz
	\end{lstlisting}
	Change directory to parsec-3.0
	\begin{lstlisting}[language=bash, xleftmargin=\dimexpr-\leftmargini]
	$ cd parsec-3.0
	\end{lstlisting}
	
	\item Setup environmental variables \\
	Setup environmental variables using following command:
	\begin{lstlisting}[language=bash, xleftmargin=\dimexpr-\leftmargini]
	$ source env.sh
	\end{lstlisting}
	
	\item Build Benchmark
	Build x264 benchmark using following command:
	\begin{lstlisting}[language=bash, xleftmargin=\dimexpr-\leftmargini]
	$ parsecmgmt -a build -p x264
	\end{lstlisting}
	where '-a' flag denotes desired action and '-p' flag specify package.
	
	\item Run Benchmark
	Run x264 benchmark with 'native' input and 4 threads:
	\begin{lstlisting}[language=bash, xleftmargin=\dimexpr-\leftmargini]
	$ parsecmgmt -a run -p x264 -i native -n 4
	\end{lstlisting}
	where '-i' flag denotes desired input and '-n' flag specify minimum number of threads.
\end{enumerate}


\begin{thebibliography}{9}
\bibitem{tutorial} C. B. a. K. Li, "The PARSEC Benchmark Suite Tutorial - PARSEC 2.0."
 (http://parsec.cs.princeton.edu/download/tutorial/2.0/parsec-2.0-tutorial.pdf).
\end{thebibliography}

\end{document}
